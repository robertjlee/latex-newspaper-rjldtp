\documentclass[a4paper]{scrartcl}

\usepackage[british]{babel}

\newcommand{\cpp}{\mbox{C\texttt{++}}}

\begin{document}

\title{Typesetting a newspaper}
\author{Robert J Lee}
\date{14\textsuperscript{th} June 2017}

\maketitle

\begin{abstract} Ths document describes a mechanism for typesetting a
	newspaper page, together with problems found and workarounds in
	\LaTeXe.
\end{abstract}

\section{Problem Statement}

We have a page of known dimensions and text to set out on that page.
The text takes the form of articles, each of which has a calcuable
length in column inches, along with a headline and other titular
information.

Each article will be supplied as a separate file on the filesystem,
which may be included into the document multiple times as needed. The
\LaTeX\ document itself will contain an environment to typeset the
newspaper, containing structered data with the articles to be
imported, along with any required metadata. The work of typeseting the
document will be performed in the closing of the environment.

We are only considering a single page at this time (newssheets). This
will be extended to allow multipage newspapers and other features in future.

The aim of the algorithm is to fill the page as completely as
possible, without overflowing the page and with as little blank space
as possible.

\section{Terminology}

A \textit{newspaper} is the document we are typesetting.

An \textit{article} is a sub-document consisting of a title, including
headline, and potentially multiple paragraphs of text. No restrictions
are placed on the content except that it is to be rendered as
\LaTeX\ source and may be parsed multiple times during rendering; it
should fill the same page area with each pass.

The \textit{editor} is the user of the program, who is assumed to have
collected the articles and to have final responsibility for passing
the finished design. This is considered to be synonymous with
\textit{layout designer} for the purposes of this document.

\section{Research}

It seems that, in the real world, newspaper editors start with a basic
(if paramaterised) laout and then commission their authors to write
articles to length, to fit the avaialble column inches given by the
template. This bypasses the complexity of having to place
arbitrary-length articles in a pleasing way.\footnote{\texttt{\tiny
    https://tex.stackexchange.com/questions/49777/automated-newspaper-layout-with-tex-and-abroad}}

One mitigation would be to allow variable-sized elements, such as
photographs, that could adjust to fill the avaialable space.


\section{Algorithm}

The plan here is to fill the page as completely as possible. Each
article will be typeset in a rectangular box, with a header containing
the centered headline across the full width, and the article set
underneath in columns.

We want to lay out the articles on the page such that no article needs
a space that is not rectangular. We will prefer filled pages to
less-filled pages --- which is to say, we want the whitespace to be in
the heading --- but we discard solutions that overflow the page ---
that is, there cannot be so much whitespace in the heading that the
text is allowed to overflow onto the second page.

\begin{enumerate}
  \item Calculate the number of column inches in each article.
    \subitem[*] This is basically a solved problem. We set the page
    width to single-column width, maximum length with no margins,
    \verb!\include! the article text, and override the \verb!\shipout!
    routine to strip rubber spaces, store the length of the article,
    and then discard the typeset page. This algorithm is run using
    \verb!\eject! to force an immediate \verb!\shipout!.
    \subitem[*] A limitation of this approach is that an article that
    will not fit on a single page will exceed the maximium page length
    and return a short count. This is generally not seen on
    newspapers, as such an article would likely fill multiple sheets.
    \subitem[*] A second limitation is that the length may be slightly
    inaccurate due to dynamic content, such as forward-references
    whose length is unknown. In this case, there may be a difference
    equal to 1--2 lines in the calculated length. In general, we
    expect \LaTeX\ to compensate for this by overfilling or
    underfilling the \verb!vbox!, and the editor can adjust this
    manually, by editing the text or overriding the length in the
    document metadata.
    \item Let $N_C$ be the whole number of columns into which the
      pages' width can be divided, $W_C$ be the width of an individual
      column
    \item Next, the vertical space of each heading is determined at
      different possible column widths. For each article:
      \subitem For $i$ from $1$ to $N_C$:
      \subsubitem Let $W_i$ be the width of $i$ columns, including alleys.
      \subsubitem Set the page width to $i \times w_i$ and repeat the
      process for measuring the article length, but this time output
      only the header.
      \subsubitem Let $L_hi$ be the length of the header at a width of $i$ columns.
    \item Now we have the length of each heading and article, let's
      try and fill the area of the page. For each artcile $a$:
      \subitem For $i$ from $1$ to $N_C$: \subitem Calculate $A_ai$,
      the area of the page taken up by the article set at this column
      width, excluding any alleys on the sides or footer of the
      article. To simplify the calculations later, this will be
      calculated in column inches, as $A_ai = \frac{L_hi}{i} + i L_ai$
    \item We then have a matrix of values $A_ai$, in column inches,
      for all articles $a$ for all possible column-widths $i$.
    \item We then choose a value of $i$ for each article $a$, such
      that the total area is as close as possible to, but not
      exceeding, the page area. We exclude from consideration any
      articles whose length at column width $i$ would be longer that
      the available area on the page.
    \item If there is no solution at this stage, output an error. The
      editor can remove or shorten articles and try the whole process
      again.
    \item So we know which articles to output and the number of
      columns we'd like. The next stage is to work out how the
      rectangles will fit onto the page. We sort the articles by the
      number of column inches, largest first.
\end{enumerate}

The simplest method to fill the page then becomes worst-fit. We keep
track of the contiguous allocated space, and fit each unplaced item on
the page, always into the largest contiguous area. By placing the
smallest articles last, the chances of success are maximised. We can
allow an article to be placed into an area that's a few points too
small, assuming a small rubber length in the interline spacing.

It's possible this may produce no result. In this case, we can try
several other algorithms:

\begin{itemize}
  \item shuffling the articles and trying to place them again,
    although the chances of success are reduced.
  \item trying a best-fit algorithm. This is very likely to work in
    the case where the editor has chosen the articles by length to fit
    the page.
\end{itemize}

It's possible this may produce no result. In this case, we need to try
a more thorough algorithm that allows for the breaking of articles.

If no result is found, we repeat the algorithm with diffing numbers of
columns in each article to adjust the space it takes up. By reducing
the number of columns, the space used by the title is reduced and so
the area required is less, although the length required increases.

In the short term, we will output an error describing the best fit
that we were able to find.

In future, we will consider other forms of layout, including splitting
articles into non-rectangular spaces and starting with rough
``templates'' (eg largest item in the centre, and wrapping articles
around it te fill the page).

\section*{Limitations}

The above approach is not guaranteed to fill the page, as the article
size is settled before placement can begin. This means that spaces may
be unavailable for the determined shapes of articles.

For a much reduced example, considering a $3 \times 3$ ($\mbox{width}
\times \mbox{height}$) grid of paper; two articles sized at $1 \times 3$ and
$3 \times 2$ would fill the area of the page exactly, but after placing the
$3 \times 2$ article, only a $3 \times 1$ row at the bottom of the page remains,
which will not acommodate the $1 \times 3$ columnar article; it would have to
be split across three columns, which would affect the space given to
the header.

In this example, it may be better to lay out the $1 \times 3$ article into
the $3 \times 1$ column and split the article onto two pages. This decision
could be done by the editor with sufficient understanding of the space
avaialble. Alternatively, the column could be split paste-up style,
with the heading constrained to the first column to preserve the
spacing. Again, it should be the editor's decision if this should be
permitted.

Another problem is where the space is available in a non-rectangular
block: Two filled articles may allow insufficient space between them
for the final article to be placed, even though sufficient contiguous
space is avaialble. In this case, a previously-placed article must be
moved to cause the space to become rectangular (if possible), or the
article may be flowed around the available space (if permitted by the
editor).

Another problem is if the space on the page is discontiguous. It is
not customary to break apart an article where it continues elsewhere
on the same page, although ``continued on page~$N$\dots'' is a common
device to increase the number of headlines on the front page. So we
must then look to move articles to free up the available space.


\section{Alternative, Template-Based Approaches}

The difficulty of any 2-dimensional layout alrogithm is that it is
difficult to determine which layouts are most aesthetically pleasing
to the reader.

This limitation can be removed by the use of parameterised layouts.
For example, a layout might stipulate that the page should allow two
single-column articles on the leftmost and rightmost columns, a
full-width image in the middle of the page, and further articles to be
fit in the remaining space on the top and bottom. The editor would
then need to provide two separate articles of the correct length for
the side columns, while the computer could work out the vertical
position of the image based on the length of the articles placed in
the top and bottom columns.

This could be achieved in a much more specific way, by allowing an
editor to specify possible templates or to choose from a list.

Alternatively, the editor could simply specify meta-data for each
article, such as an affinity for a particular place on the page, an
override for the number of columns, and/or an order for placement on
the page. In this way, the editor is still asking the computer to lay
out the page, but has far more control over the final outcome; this
moves some of the responsibility for finding a correct and aesthetic
design to the editor.


It seems likely that a number of approaches could be used by a
pluggable system of algorithms.

\section{Problem: Breaking Articles into columns}

This is a solved problem, but not as simple as described above.

If the article is rectangular then it can be done simply by setting
the article inside a balanced \verb!multicols! environment.

Otherwise, there is a solution that involves modifying the input text
to avoid the implicit horizontal space at the end of the paragraph; I
wrote the \verb!rjldtp! package specifically to address this issue by
injecting \TeX\ primitives.

In practise, the \verb!rjldtp!-style approach is only needed when the
width of the columns changes mid-article; an element is injected into
the source text at the point where the article splits which exits
horizontal (paragraph-building) mode, inserts a column-break,
changes the \verb!\hsize! to which the text is typeset, and injects
\verb!\noindent! to avoid starting a new visual paragraph. If the
width of the column does not change, then the article can be typeset
into a box register, then that register split using \verb!\vsplit! to
the desired length. This is much simpler code, although ideally we
should do something similar to the \verb!multicols! environment to
balance the text across columns in the case where the article is short
(as \TeX\ doesn't always break boxes exactly where you want it to).

\section*{\cpp Proof-of-Concept Program}

The \cpp proof-of-concept program\footnote{Distributed in the
  ``layouttest'' directory.} was written to allow faster
experimentation with layout algorithms than is possible by programming
directly in \TeX.

The \texttt{Page} class contains metadata for the page being typeset.
\texttt{Article} contains basic metadata for an article, including
size information.
\texttt{layout::worstFit} supplies a basic worst-fit algorithm.

As this document is written, the \texttt{main} method will add
articles to the document, with each article having options for 1--5
columns. We then calculate all permutations of all article options in
order to determine the permutation that best fills the page without
overflow; this is by far the slowest part of the program.

The worst-fit algorithm is a simple implementation that keeps track of
rectangular areas of free space, which are treated independently. If
no solution can be found, backtracking is performed not on the fitment
of pages onto the layout but on the permutations of the articles on
the document.

In practise, this appears to work fast enough in most cases.

This could easily be improved by:

\begin{itemize}
\item Backtracking in article fitment (ie trying each article in the
  second-worst placement and seeing if this improves the situation);
\item Backtracking in the method by which free space is divided (this
  is easy to do but not thorough)
\item Looking at all corners for placement, not just the top-left
  corner of the article placement;
\item Considering ways to typeset articles that overflow more than one
  adjacent free space (this would be more thorough but negate the
  benefits of backtracking in the method by which free space is
  divided).
\end{itemize}

In addition, affinity for a particular place could be included to
particular articles, especially a main headline to be set top-left.

The program makes no attempt to typeset headers and footers. These are
assumed to be outwith the scope of the program, which deals only with
the algorithm.

Alleys are also ignored; these affect the algorithm in theory as they
take space on the page; however, the space can generally be included
in the size of the article, so any deviation from a perfect solution is
expected to be small; therefore, the expense in time and program
complexity of implementing alleys is not justified.

\section*{Alley placement}

Alleys are lines drawn between articles to separate them.

Determining where alleys should be placed is not entirely trivial,
because the alley lines do not draw to the full extent of the article size.

It is hereby hypothesised that a alley should be interrupted by a gap
at any corner of an article, regardless of the side of the article on
which the alley resides. Therefore, it is not sufficient merely to
place an article on the page, place its alleys and then place the
adjacent articles; the size and length of the adjacent articles must
be known when the alley is placed.

\pagebreak[1]

\begin{verbatim}
  1234567
1 +-+-+-+
2 |A|B|C|
3 +-+ | |
4 | +-+ |
5 |D|E+-+
6 | +-+ |
7 | |F|G|
8 +-+-+-+
\end{verbatim}

This shows six articles in an offset $3\times 3$ grid.
The ideal alleys would be:

\begin{tabular}{r|c|l}
Column & 1 & 2, 4--7 \\
Column & 3 & 2, 5, 7 \\
Column & 5 & 2--3, 4.5, 5.5, 7 \\
Column & 7 & 2--4, 6--7 \\
Row & 1 & 2, 4, 6 \\
Row & 3 & 2 \\
Row & 4 & 4 \\
Row & 5 & 6 \\
Row & 8 & 2, 4, 6
\end{tabular}

\section*{Thoughts on Further Development}

Given the current state of the \cpp program, there are two different
methods that could be employed going forward.

The current program emits the paper coordinates for each article; we
use this to typeset the articles in exact locations on the page. This
has the advantage of more easily allowing for non-rectangular article
layout in the future, but the disadvantage of having to calculate the
exact positions of the alleys.

The current algorithm works by dividing the page into rectilinear
areas. It would be trivial to expose this part of the algorithm and
emit a horizontal or vertical alley at each point where the
whitespace is broken; we know at this point the size of the alley and
whether there is another article from which to separate above or below
the alley itself. This has the advantage of improving the layout for
rectilinear articles, as rubber space can be used to allow the
\TeX\ engine to optimally lay out the page. It has the disadvantage
that this approach would be harder to generalise to non-rectilinear
article layout.

Even if we can preserve the encapsulation of the first method, with
all the many benefits that encapsulation brings for maintainability,
the fact that we could improve the page layout in the second method
suggests that it should be retained as an option
nonetheless. Unfortunately, if lines are to be placed in the alley
then the length and size of all adjacent articles must be known before
the alleys can be placed, which negates any optimisation obtained by
this method.

Thus type-safety wins again; a pluggable algorithm that outputs the
article locations will be used, and the alleys will be retrofitted.

\section*{Further Reading}

ARC project\footnote{\texttt{\tiny
    http://cgi.csc.liv.ac.uk/~epa/surveyhtml.html}} has a number of
rectangle-packing algorithms.

\end{document}
