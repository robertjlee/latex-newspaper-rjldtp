\documentclass[a4paper]{scrartcl}

\usepackage[british]{babel}

\newcommand{\cpp}{\mbox{C\texttt{++}}}

\begin{document}

\title{Typesetting a newspaper}
\author{Robert J Lee}
\date{14\textsuperscript{th} June 2017}

\maketitle

\begin{abstract} 
  Manual for a program for typesetting a newspaper page.
\end{abstract}


\section{Terminology}

A \textit{newspaper} is the document we are typesetting.

An \textit{article} is a sub-document consisting of a title, including
headline, and potentially multiple paragraphs of text. No restrictions
are placed on the content except that it is to be rendered as
\LaTeX\ source and may be parsed multiple times during rendering.

The \textit{editor} is the user of the program, who is assumed to have
collected the articles and to have final responsibility for passing
the finished design. This is considered to be synonymous with
\textit{layout designer}


\section{Description of System}

\subsection{Architecture}

The system consists of a \cpp program to perform page layout and
coordinate the output. Input is taken from article files, one per
source file, and a metadata file. \LaTeXe\ is used variously to parse
the input files, determine the possible dimensions of each article and
to typeset the final page.


\subsection{Limitations}

This system may fail to produce an output page, if there is too much
text to fit on the page or if articles cannot be laid out without
overlapping.

The current algorithm does not perform an exhaustive search, as this
would take an amount of time exponentionally proportional to the
number of articles, so some theoretically valid cases may also not
produce a complete page of output.

Multiple pages are not yet implemented; this is considered for a
future version.

No attempt is made to balance the page or deal with extra whitespace.

Articles may be set in multiple columns, but will always be laid out
on a rectangle. There are no cutaways for images or to wrap around
other articles.

While any \LaTeX\ code may be used in the articles, the articles
themselves may be included and parsed muliple times. Therefore, care
must be taken with anything more complex than a self-contained
document section; for example, when incrementing the value of a
counter, the incrementation may happen repeatedly as the possible
dimensions of the document are worked out and again when the final
document is typeset.

Article text will be read as \LaTeX\ source code. This provides a
great deal of flexibility, but means that some preprocessing may be
needed if the articles are supplied in other formats, such as plain
text, markdown, or the format of a WYSIWYG editor.

\TeX\ and \LaTeX\ require special consideration when handling text in
narrow columns; generally, the standard of full-justified layout
required by default is too high to be rendered as the defaults are
intended for full-width paperbacks and similar articles.

\section{Input File Format}

An example input file named \texttt{paper.tex} looks like this:

\begin{verbatim}
\documentclass{rjlnewsp4}
\usepackage[paperheight=10.75in,paperwidth=8.25in,margin=1in]{geometry}
\usepackage[british]{babel}
\usepackage{indentfirst}
\begin{document}
\begin{newspaper}
  \article{article1.art}{Headline 1}[\hfill Mx Smith]
  \article{article2.art}{Headline 2}[Mx Smith]
  \article{article3.art}{Headline 3}
  \article{article4.art}{Headline 4}[Mx Smith]
\end{newspaper}
\end{document}
\end{verbatim}

This structure may be extended with further options in future.

\begin{description}
  \item[documentclass] \hfill This is the \LaTeX\ class of the input
    document. It must be the first line in the file.
  \item[geometry] \hfill This \LaTeX\ package is one of many ways of
    specifying the paper size and outer margins.
  \item[babel] \hfill While optional, it is strongly recommended to
    include the \verb!babel! package, which sets up hyphenation rules,
    essential for typesetting words in narrow columns. The
    \verb!british! option changes the hypenation rules from American
    English to British; see the Babel package documentation for more
    details and further options.
  \item[indentfirst] \hfill Without this package, the first paragraph
    of each article is unindented; newspapers typically indent the
    first paragraph.
  \item[document] \hfill The \verb!\begin{document}! and
    \verb!\end{document}! lines bookend the body of any
    \LaTeX\ document. Any additional \LaTeX\ modules used by the
    articles should be included before this point.
  \item[newspaper] \hfill The \verb!\begin{newspaper}! and
    \verb!\end{newspaper}! lines create the \texttt{newspaper}
    environment, in which articles are typeset as a newspaper. Adding
    any additional content outside this environment may confuse the
    \cpp\ program and cause layout errors.
  \item[\texttt{$\backslash$ article}] \hfill This declares an individual article to
    be typeset on this page. This command has three parameters:
    \begin{description}
      \item[\texttt{article1.art}] \hfill This is the name of a file
        containing the article text.
      \item[\texttt{Headline 1}] \hfill This is the article headline.
        Note that longer headlines will take more vertical space when
        multi-column articles are used.
      \item[\texttt{Mx Smith}] \hfill It is suggested that a
        sub-headline may be used for the author's name. \verb!\hfill!
        can be used before the sub-headline to right-align it, or after the
        name to left-align it. The sub-headline may be omitted
        completely as shown in the third article.
    \end{description}
\end{description}

Each article file contains only the article text, without headers,
footers or other ornamentation.

The text is to be supplied in plain \LaTeX\ source code, without any
preamble, postable or headers. This gives certain characters special
meaning, and not all article files are legal. To convert from plain
text to \LaTeX\ the following is usually sufficient, although
additional changes may be needed if other packages or commands are
declared before \verb!\begin{document}!:

\begin{description}
  \item{\$ \~ \% \_ \{ \} \# } These special characters must be escaped
   with a leading backslash~(\textbackslash).
  \item{"} While legal, the double-quote character will be typeset as
    an inch sign. Use backticks and single-quote marks for opening and
    closing quotes, which may be doubled. \verb!``!For example\verb!''!
    will be shown as ``For example''.
  \item{\textless \textgreater \textbackslash} should be set as
    \textbackslash textless, \textbackslash textgreater or
    \textbackslash textbackslash, followed by a space which is not
    printed. If you want the space to be printed, you can follow it
    with \textbackslash then space instead.
\end{description}

In addition, you should leave a blank line between each paragraph.

As a special rule, if you use accented characters in the article, you
should start with the line:

\begin{verbatim}
\inputencoding{utf8}
\end{verbatim}

You may need to change \verb!utf8! to match the encoding of the file,
eg to \verb!latin1! for ISO~Latin-1.

\end{document}
